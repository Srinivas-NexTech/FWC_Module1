\documentclass[12pt]{article}
\usepackage[T1]{fontenc}
\usepackage{graphicx}
\usepackage[a4paper,margin=1in]{geometry} % Page layout
\usepackage{amsmath,amssymb} % Math symbols
\usepackage{enumitem}        % Custom lists
\usepackage{fancyhdr}        % Headers/footers
\usepackage{titlesec}  
\date{\tdoady}
\geometry{top=1in, bottom=1in, left=1in, right=1in}    % Header and Footer settings
\pagestyle{empty}
\begin{document}
\thispagestyle{fancy}
\fancyhf{} % Clear default header and footer
\fancyhead[L]{% Left header
	\includegraphics[width=8cm, height=1.7cm]{logo.png} % Adjust dimensions
}
\fancyhead[R]{% Right header
    Name: Namathoti Srinivas \\
    Batch: COMETFWC036 \\
    Date: 08 August 2025}
\renewcommand{\headrulewidth}{0pt} % Remove header line
\fancyfoot[C]{\thepage} % Page number centered in footer
\vspace*{1cm}
\textbf{Nineteenth International Mathematical Olympiad, 1977}
\section*{1977/1}
Equilateral triangles $ABK$, $BCL$, $CDM$, $DAN$ are constructed inside the square $ABCD$.  
Prove that the midpoints of the four segments $KL$, $LM$, $MN$, $NK$ and the midpoints of the eight segments $AK$, $BK$, $BL$, $CL$, $CM$, $DM$, $DN$, $AN$ are the twelve vertices of a regular dodecagon.

\section*{1977/2}
In a finite sequence of real numbers, the sum of any seven successive terms is negative,  
and the sum of any eleven successive terms is positive.  
Determine the maximum number of terms in the sequence.

\section*{1977/3}
Let $n$ be a given integer $> 2$, and let $V_n$ be the set of integers $1 + kn$, where $k = 1, 2, \dots$.  
A number $m \in V_n$ is called indecomposable in $V_n$ if there do not exist numbers $p, q \in V_n$ such that $pq = m$.  
Prove that there exists a number $r \in V_n$ that can be expressed as the product of elements indecomposable in $V_n$ in more than one way.  
(Products which differ only in the order of their factors will be considered the same.)

\section*{1977/4}
Four real constants $a, b, A, B$ are given, and
$
f(\theta) = 1 - a \cos \theta - b \sin \theta - A \cos 2\theta - B \sin 2\theta.
$
Prove that if $f(\theta) \ge 0$ for all real $\theta$, then
$
a^2 + b^2 \le 2 \quad \text{and} \quad A^2 + B^2 \le 1.
$

\section*{1977/5}
Let $a$ and $b$ be positive integers.  
When $a^2 + b^2$ is divided by $a + b$, the quotient is $q$ and the remainder is $r$.  
Find all pairs $(a, b)$ such that $q^2 + r = 1977$.

\section*{1977/6}
Let $f(n)$ be a function defined on the set of all positive integers and having all its values in the same set.  
Prove that if
$
f(n + 1) > f(f(n))
$
for each positive integer $n$, then
$
f(n) = n
$
for each $n$.




\newpage\textbf{Twentieth International Olympiad, 1978}

\section*{1978/1}
$m$ and $n$ are natural numbers with $1 \le m < n$.  
In their decimal representations, the last three digits of $1978^m$ are equal, respectively, to the last three digits of $1978^n$.  
Find $m$ and $n$ such that $m + n$ has its least value.

\section*{1978/2}
$P$ is a given point inside a given sphere.  
Three mutually perpendicular rays from $P$ intersect the sphere at points $U, V,$ and $W$;  
$Q$ denotes the vertex diagonally opposite to $P$ in the parallelepiped determined by $PU, PV,$ and $PW$.  
Find the locus of $Q$ for all such triads of rays from $P$.

\section*{1978/3}
The set of all positive integers is the union of two disjoint subsets  
$\{ f(1), f(2), \dots, f(n), \dots \}$ and $\{ g(1), g(2), \dots, g(n), \dots \}$,  
where
$
f(1) < f(2) < \dots < f(n) < \dots,
$
$
g(1) < g(2) < \dots < g(n) < \dots,
$
and
$
g(n) = f(f(n)) + 1 \quad \text{for all } n \ge 1.
$
Determine $f(240)$.

\section*{1978/4}
In triangle $ABC$, $AB = AC$.  
A circle is tangent internally to the circumcircle of triangle $ABC$ and also to sides $AB$ and $AC$ at $P, Q$ respectively.  
Prove that the midpoint of segment $PQ$ is the center of the incircle of triangle $ABC$.

\section*{1978/5}
Let $\{ a_k \}$ ($k = 1, 2, 3, \dots, n, \dots$) be a sequence of distinct positive integers.  
Prove that for all natural numbers $n$,
$
\sum_{k=1}^n \frac{a_k}{k^2} \ge \sum_{k=1}^n \frac{1}{k}.
$

\section*{1978/6}
An international society has its members from six different countries.  
The list of members contains $1978$ names, numbered $1, 2, \dots, 1978$.  
Prove that there is at least one member whose number is the sum of the numbers of two members from his own country,  
or twice as large as the number of one member from his own country.
 
\newpage\textbf{\large Twenty-first International Olympiad, 1979}

\titleformat{\section}{\normalfont\large\bfseries}{}{0.5em}{}
 
\section*{1979/1}
Let $p$ and $q$ be natural numbers such that
$p/q = 1 - 1/2 + 1/3 - 1/4 + \cdots - 1/1318 + 1/1319$.
Prove that $p$ is divisible by $1979$.
\section*{1979/2}
A prism with pentagons $A_1A_2A_3A_4A_5$ and $B_1B_2B_3B_4B_5$ as top and bottom faces is given.  
Each side of the two pentagons and each of the line segments $A_iB_j$ for all $i,j=1,\dots,5$ is colored either red or green.  
Every triangle whose vertices are vertices of the prism and whose sides have all been colored has two sides of a different color.  
Show that all $10$ sides of the top and bottom faces are the same color.
\section*{1979/3}
Two circles in a plane intersect. Let $A$ be one of the points of intersection.  
Starting simultaneously from $A$ two points move with constant speeds, each point travelling along its own circle in the same sense.  
The two points return to $A$ simultaneously after one revolution.  
Prove that there is a fixed point $P$ in the plane such that, at any time, the distances from $P$ to the moving points are equal.
\section*{1979/4}
Given a plane $\pi$, a point $P$ in this plane and a point $Q$ not in $\pi$, find all points $R$ in $\pi$ such that the ratio $(QP + PA)/QR$ is a maximum.
\section*{1979/5}
Find all real numbers $a$ for which there exist non-negative real numbers $x_1, x_2, x_3, x_4, x_5$ satisfying the relations
$\sum_{k=1}^{5} kx_k = a$,  
$\sum_{k=1}^{5} k^3 x_k = a^2$,  
$\sum_{k=1}^{5} k^5 x_k = a^3$.
\section*{1979/6}
Let $A$ and $E$ be opposite vertices of a regular octagon.  
A frog starts jumping at vertex $A$. From any vertex of the octagon except $E$, it may jump to either of the two adjacent vertices.  
When it reaches vertex $E$, the frog stops and stays there. Let $a_n$ be the number of distinct paths of exactly $n$ jumps ending at $E$.  
Prove that $a_{2n-1} = 0$,  
$a_{2n} = \frac{1}{\sqrt{2}}\,(x^{n-1} - y^{n-1})$, $n=1,2,3,\dots$  
where $x = 2 + \sqrt{2}$ and $y = 2 - \sqrt{2}$.
\medskip
\noindent
\textit{Note.} A path of $n$ jumps is a sequence of vertices $(P_0, \dots, P_n)$ such that:
\begin{enumerate}[label=(\roman*)]
    \item $P_0 = A$, $P_n = E$;
    \item for every $i$, $0 \le i \le n-1$, $P_i$ is distinct from $E$;
    \item for every $i$, $0 \le i \le n-1$, $P_i$ and $P_{i+1}$ are adjacent.
\end{enumerate}

\end{document}

