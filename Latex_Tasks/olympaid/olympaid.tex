\documentclass[a4paper,12pt]{article}
\usepackage{amsmath, amssymb, xcolor, tcolorbox}
\usepackage{geometry}
\usepackage{enumitem}
\usepackage{fancyhdr}
\usepackage{graphicx}
\geometry{top=1in, bottom=1in, left=1in, right=1in}    % Header and Footer settings
\pagestyle{plain}

\begin{document}

\thispagestyle{fancy}
\fancyhf{} % Clear default header and footer
\fancyhead[L]{% Left header
        \includegraphics[width=8cm, height=1.7cm]{IIITB-COMET-Logo.png} % Adjust dimensions
}
\fancyhead[R]{% Right header
    Name: N.SRINIVAS \\
    Batch: COMETFWC036 \\
    Date: 11 August 2025
}
\renewcommand{\headrulewidth}{0pt} % Remove header line
\fancyfoot[C]{\thepage} % Page number centered in footer
\vspace*{0.1em}
\section*{\textbf{\Large Thirteenth International Olympiad, 1977}}
\begin{itemize}[label={},leftmargin=0pt]
\item \textbf{\large 1977/1.}
\item Prove that the following assertion is true for $n = 3$ and $n = 5$, and that it is false for every other natural number $n > 2$ :
\item If $ a_1, a_2, ..., a_n$ are arbitrary real numbers, then 
\item $(a_1 - a_2)(a_1 - a_3)\cdot \cdot \cdot (a_1 - a_n) + (a_2 - a_1)(a_2 - a_3)\cdot \cdot \cdot  (a_2 - a_n) + \cdot \cdot \cdot + (a_n - a_1)(a_n - a_2)\cdot \cdot \cdot (a_n - a_{n-1}) \geq 0$
\item \textbf{\large 1977/2.}
\item Consider a convex polyhedron $P_1$ with nine vertices $A_1,A_2, ..., A_9$; let $P_i$ be the polyhedron obtained from $P_1$ by a translation that moves vertex $A_1$ to$A_i(i = 2, 3, ..., 9)$. Prove that at least two of the polyhedra $P_1, P_2, ..., P_9$ have an interior point in common.
\item \textbf{\large 1977/3.}
\item Prove that the set of integers of the form $2^k - 3(k = 2, 3, ...)$ contains an infinite subset in which every two members are relatively prime.
\item \textbf{\large 1977/4.}
\item All the faces of tetrahedron $ABCD$ are acute-angled triangles. We consider all closed polygonal paths of the form $XY ZT X$ defined as follows: $X$ is a point on edge AB distinct from $A$ and $B$; similarly, $Y, Z, T$ are interior points of edges $BCCD, DA$, respectively. Prove:
\begin{enumerate}[label=(\alph*)]
	\item If $\angle DAB + \angle BCD \neq \angle CDA + \angle ABC$, then among the polygonal paths, there is none of minimal length.
	\item If $\angle DAB + \angle BCD = \angle CDA + \angle ABC$, then there are infinitely manyshortest polygonal paths, their common length being $2AC \sin (\alpha /2)$, where $ \alpha = \angle BAC + \angle CAD +  \angle DAB$.
\end{enumerate}
\item \textbf{\large 1977/5.}
\item Prove that for every natural number $m$, there exists a finite set $S$ of points in a plane with the following property: For every point $A$ in $S$, there are exactly $m$ points in $S$ which are at unit distance from $A$.
\item \textbf{\large 1977/6.}
\item Let $A = (a_{ij} )(i, j = 1, 2, ..., n)$ be a square matrix whose elements are non-negative integers. Suppose that whenever an element $a_{ij} = 0$, the sum of the elements in the $i^{th}$ row and the $j^{th}$ column is $\geq n$. Prove that the sum of all the elements of the matrix is $\geq n^2/2$.
\end{itemize}
\section*{\textbf{\Large Fourteenth International Olympiad, 1978}}
\begin{itemize}[label={},leftmargin=0pt]
	 \item \textbf{\large 1978/1.}
	 \item Prove that from a set of ten distinct two-digit numbers (in the decimal system), it is possible to select two disjoint subsets whose members have the same sum.
	 \item \textbf{\large 1978/2.}
	 \item Prove that if $n \geq 4$, every quadrilateral that can be inscribed in a circle can be dissected into $n$ quadrilaterals each of which is inscribable in a circle.
	 \item \textbf{\large 1978/3.}
	 \item Let $m$ and $n$ be arbitrary non-negative integers. Prove that
	\begin{center}	 
		$\dfrac{(2m)!(2n)!}{m!n!(m + n)!}$
	\end{center}
	\item is an integer. $(0! = 1.)$
	\item \textbf{\large 1978/4.}
	\item Find all solutions $(x_1, x_2, x_3, x_4, x_5)$ of the system of inequalities
	\begin{center}
$(x_2^1 - x_3 x_5)(x_2^2 - x_3 x_5)  \leq  0$ \\
$(x_2^2 - x_4 x_1)(x_2^3 - x_4 x_1)  \leq  0$ \\
$(x_2^3 - x_5 x_2)(x_2^4 - x_5 x_2)  \leq  0 $ \\
$(x_2^4 - x_1 x_3)(x_2^5 - x_1 x_3)  \leq  0 $ \\
$(x_2^5 - x_2 x_4)(x_2^1 - x_2 x_4)  \leq  0 $

	\end{center}
\item where $x_1, x_2, x_3, x_4, x_5 $ are positive real numbers
\item \textbf{\large 1978/5.}
\item Let $f$ and $g$ be real-valued functions defined for all real values of $x$ and $y$, and satisfying the equation
$f(x + y) + f(x - y) = 2f(x)g(y)$
for all $x$, $y$. Prove that if $f(x$) is not identically zero, and if $|f(x)| \leq 1$ for all $x$, then $|g(y)| \leq 1$ for all $y$.
\item \textbf{\large 1978/6.}
\item Given four distinct parallel planes, prove that there exists a regular tetrahedron with a vertex on each plane.
\end{itemize}
\newpage
\section*{\textbf{\Large Fifteenth International Olympiad, 1979}}
\begin{itemize}[label={},leftmargin=0pt]
\item \textbf{\large 1979/1.}
\item Point $O$ lies on line $g$; $\vec{OP_1}, \vec{OP_2}, ..., \vec{OP_n}$ are unit vectors such that points $P_1, P_2, ..., P_n$ all lie in a plane containing $g$ and on one side of $g$. Prove that if $n$ is odd,
	\begin{center}
		$|\vec{OP_1} +\vec{OP_2} + \cdot \cdot \cdot + \vec{OP_n}| \geq 1$
	\end{center}
		Here $|\vec{OM}|$ denotes the length of vector $\vec{OM}$.
\item \textbf{\large 1979/2.}
\item Determine whether or not there exists a finite set $M$ of points in space not lying in the same plane such that, for any two points $A$ and $B$ of $M$, one can select two other points $C$ and $D$ of $M$ so that lines $AB$ and $CD$ are parallel and not coincident.
\item \textbf{\large 1979/3.}
\item Let $a$ and $b$ be real numbers for which the equation
	\begin{center}
		$x^4 + ax^3 + bx^2 + ax + 1 = 0$
	\end{center}
has at least one real solution. For all such pairs $(a, b)$, find the minimum value of $a^2 + b^2$.
\item \textbf{\large 1979/4.}
\item A soldier needs to check on the presence of mines in a region having the shape of an equilateral triangle. The radius of action of his detector is equal to half the altitude of the triangle. The soldier leaves from one vertex of the triangle. What path shouid he follow in order to travel the least possible distance and still accomplish his mission?
\item \textbf{\large 1979/5.}
\item $G$ is a set of non-constant functions of the real variable $x$ of the form
	\begin{center}
		$f(x) = ax + b$, $a$ and $b$ are real numbers,
	\end{center}
and $G$ has the following properties:
\begin{enumerate}[label=(\alph*)]
	\item If $f$ and $g$ are in $G$, then $g \circ f$ is in $G$; here $(g \circ f)(x) = g[f(x)]$.
	\item If $f$ is in $G$, then its inverse $f^{-1}$ is in $G$; here the inverse of $f(x) = ax+b$ is $f^{-1}(x) = (x - b)/a$.
	\item For every $f$ in $G$, there exists a real number $x_f$ such that $f(x_f ) = x_f$ .
\end{enumerate}
	\item Prove that there exists a real number $k$ such that $f(k) = k$ for all $f$ in $G$.
\item \textbf{\large 1979/6.}
\item Let $a_1, a_2, ..., a_n$ be $n$ positive numbers, and let $q$ be a given real number such that $0 < q < 1$. Find $n$ numbers $b_1, b_2, ..., b_n$ for which
\begin{enumerate}[label=(\alph*)]
	\item $a_k < b_k$ for $k = 1, 2, ... , n,$
	\item $q < \frac{b_k+1}{b_k} < \frac{1}{q}$ for $k = 1, 2, ..., n - 1$,
	\item $b_1 + b_2 + \cdot \cdot \cdot + b_n < \frac{1+q}{1-q}(a_1 + a_2 + \cdot \cdot \cdot + a_n).$
\end{enumerate}
\end{itemize}
\end{document}
